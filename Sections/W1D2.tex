% !TeX root = ../main.tex
\section{Cohomology of Groups}
\label{sec:W1D2}

\begin{definition}
    \label{def:n-connected}
    We say a space \(Y,y_0\) is \emph{\(n\)-connected} if \(\pi_i(Y,y_0) =0\) for all \(0\leq i \leq n\).
\end{definition}

Other than the trivial coefficient \(\Z\), we use also group rings to calculate the cohomology of groups to get different properties. 

\begin{definition}
    \label{def:group-ring}
    Let \(G\) be a group. The group ring \(\Z G\) is the set of finite \(Z\)-linear combinations of elements of \(G\).
\end{definition}

\begin{remark}
    We may also use \(\R\),  fields, or PIDs, etc. instead of \(\Z\) to get different group rings.
\end{remark}

\begin{example}
    Let \(G=\Z =\langle t \rangle \). Then \(\Z G =  \Z [t,t^{-1}]\).
\end{example}

\begin{example}
    Let \(G=\Z/n\Z = \langle t \rangle \). Then \(\Z G= \Z[t]/(t^n-1)\).
\end{example}



Notice that \(g^n  =1\) for some non-trivial element \(g\in G\), i.e. \(G\) has a non-trivial torsion, then we have: 
\[ 0 = g^n -1 = (g-1)(g^{n-1}+\cdots + g+1) \]
which implies that \(g-1\) is a non-zero zero-divisor. The other direction of this fact is an open question.

\begin{question}[Kaplansky's Conjecture]
    \label{conj:torsion-free-no-zero-divisor}
    If \(G\) is torsion-free, then \(\Z G\) has no non-zero zero-divisor.
\end{question}

A motivation for \cref{conj:torsion-free-no-zero-divisor} is that we may embed \(\Z G\) into its division ring, so we wish to have no non-zero zero-divisors in \(\Z G\).


Typically \(G\) is not abelian, so \(\Z G\)is not a commutative ring. Left and right \(\Z G\)-modules are typically different. By default, we use left \(\Z G\)-modules.

\begin{remark}
    \label{rmk:left-right-module-transfer}
    If \(M\) is a left \(\Z G\)-module, it can be turned into a right \(\Z G\)-module by defining a right \(G\)-action \(mg = g^{-1}m\).
\end{remark}

\begin{definition}
    \label{def:resolution-free-ZG-module}
    A free resolution of \(\Z\) (the trivial \(\Z G\)-module) by free \(\Z G\)-module is a resolution
    \[ \cdots\to F_1 \to F_0 \to \Z \to 0\]
    that is exact, and every \(F_i\) is a free \(\Z G\)-module (i.e. \((\Z G)^n\) for some \(n\)).

    We sometimes use also resolutions by projective \(\Z G\)-modules.
\end{definition}

\begin{remark}
    \label{rmk:free-resolution-arises-naturally}
    The free resolution arises naturally from \(K(G,1)\) in the following manner. Say \(X\) is a CW complex \(K(G,1)\) (recall \cref{def:KG1}), then any \(i\)-cell in \(\tilde X\) is \(g\sigma\) for some \(g\in G\) and \(\sigma \) an \(i\)-cell in \(X\). So we can view the cellular \(i\)-chain \(C_i(\tilde X; \Z)\) as a free \(\Z G\)-module with basis in bijective with \(i\)-cells in \(X\). 
    % (Recall that \(G\) acts freely on \(\tilde X\), so \(G\) acts on \(C_i(\tilde X ; \Z)\) freely.)

    Since \(\tilde X\) is contractible, \(H_*(\tilde X) =0\). That is, the complex
    \[
    \cdots \to C_{n+1} (\tilde X) \xrightarrow{\partial_{n+1}} C_n \xrightarrow{\partial_n} \cdots \xrightarrow{\partial_1} C_0(\tilde X) \xrightarrow{\epsilon} \Z \to 0
    \]
    is exact. The boundary maps are \(\Z G\)-module homomorphisms. This gives a free resolution of \(\Z\) by free \(\Z G\)-modules.
\end{remark}

TODO picture example 


\begin{theorem}
    \label{thm:unique-resolution-up-to-homotopy}
    There is a unique resolution of \(\Z\) by free (projective) \(\Z G\)-module.
\end{theorem}


\begin{definition}
    \label{def:cohomology-of-group}
    Let \(M\) be a left \(\Z G\)-module. Let \(F = \{(F_i, d_i)\}\) be a free resolution of \(\Z\). We truncate \(F\) to get a (non-exact) complex:
   
     \[ 
       \cdots \rightarrow F_i \xrightarrow{d_i} F_{i-1} \xrightarrow{d_{i-1}}\cdots \xrightarrow{d_1} F_0 \to 0 .
    \]

    Apply \(\Hom{\_}{M}{\Z G}\) to the complex. Denote \(F^i = F_i^* =  \Hom{F_i }{M }{\Z G}\). We have:
    \[ 
       \cdots \leftarrow F^i \xleftarrow{d^{i-1}} F^{i-1} \xleftarrow{d^{i-2}}\cdots \xleftarrow{d^0} F^0 \xleftarrow{} 0 .
    \]

    The \emph{cohomology of \(G\)} with coefficients in \(M\), denoted as \(H^*(G;M)\), is the homology of the complex above: 
    \[ H^i(G;M) = \ker d^i/\im d^{i-1} .\]
\end{definition}

If we do not truncate the resolution, then we get the exact sequence:
\[
    \cdots \leftarrow \Hom{F_1 }{M }{\Z G} \xleftarrow{d^0 =d_1^*} \Hom{F_0}{M }{\Z G} \xleftarrow{\epsilon^*} \Hom{\Z }{M }{\Z G} \xleftarrow 0.
\]
Denote \(H^0(G;M)\) as \(M^G\). Notice that \(M^G = \ker d^0 = \im \epsilon^* \cong \Hom{\Z }{M }{\Z G}\). Recall that \(G\) acts on \(\Z\) trivially, so elements in \(\Hom{\Z }{M }{\Z G}\) has to be elements in \(M\) that are invariant under \(G\), i.e. \(M^G = \{m\in M \mid \forall g \in G, gm=m\}\).

Since free resolutions  are chain homotopic, \(H^*(G;M)\) is independent of the choices of the free resolution. 

\begin{example}
    \label{ex:cohomology-of-ZnZ}
    Let \(G=\Z/n\Z\), \(M=\Z\). Then \(\Z G = \Z[t]/(t^n-1)\). Consider \(\Z G \xrightarrow{\epsilon} \Z \to 0\), \(\ker \epsilon = (t-1)\). Let \(N = t^{n-1}+\dots + t + 1\). We have the free resolution:
    \[ 
        \cdots \to \Z G
        \xrightarrow{\cdot N }  \Z G 
        \xrightarrow{\cdot(t-1)} \Z G
        \xrightarrow{\cdot N }  \Z G 
        \xrightarrow{\cdot(t-1)}  \Z G 
        \xrightarrow{\epsilon} \Z \to 0.
    \]
\end{example}

Apply \(\Hom{\_ }{\Z }{\Z G}\)to the resolution. For any \(f\in \Hom{\Z G }{\Z }{\Z G}\), \(f(\sum_i \a_i t^i)\) has to be \(k \sum_i \a_i\) where \(k\in \Z\), i.e. \(\Hom{\Z G }{\Z }{\Z G} = \Z\). Thus the cochain complex we obtained is:
\[
    \cdots \xleftarrow{\cdot n} \Z \xleftarrow{\cdot 0} \Z  \xleftarrow{\cdot n} \Z \xleftarrow{\cdot 0} \Z \leftarrow 0.
\]

We conclude that 
\begin{equation*}
    H^i(\Z/n\Z;\Z) = 
    \begin{cases}
        \Z & \text{if } i=0 ,\\
        \Z/n\Z & \text{if } i\geq 1 \text{ is odd}, \\
        0 & \text{if } i\geq 1 \text{ is even}.
    \end{cases}
\end{equation*}

\begin{proposition}
    \label{prop:high-cohomology-zero}
    If \(X\) is an \(n\)-dimensional \(K(G,1)\), \(M\) is any \(\Z G\)-module, then \(\forall m > n\), \(H^m(G;M) =0\).
\end{proposition}
That's because the free resolution is trivial for \(m>n\).

\begin{corollary}
    \label{cor:KG1-ZnZ-infinite-dimensional}
    There does not exist any finite-dimensional \(K(\Z/n\Z, 1)\) for \(n\geq 2\).
\end{corollary}

We are now ready to prove \cref{thm:torsion-criterion-KG1}. Recall the statement:

\TorsionCriterion*

\begin{proof}
    Let \(X\) be a \(K(G,1)\). For any subgroup \(H\leq G\), let \(X^H\) be the cover of \(X\) corresponding to \(H\). Then \(X^H\) is a \(K(H,1)\). Observe that \(X\) and \(X^H\) have the same dimension.

    If \(G\) has non-trivial torsion, then there exists \(\Z/n\Z \leq G\) with \(n\geq 2\). By \cref{cor:KG1-ZnZ-infinite-dimensional}, \(X^H\) is infinite-dimensional, hence so is \(X\).
\end{proof}