% !TeX root = ../main.tex
\section{Aspherical Spaces and Sample Early Theorems}
\label{sec:W1D1}

In this course we will explore the relationship between aspherical spaces and (discrete) groups. A convention: unless otherwise stated, all spaces in this course will be CW complexes.

\begin{definition}[\(K(G,1)\)]
    \label{def:KG1}
    Let \(G\) be a group.
    A \emph{\( K(G,1) \) }is a connected space such that \(\pi_1(K(G,1)) \cong G\)  and  whose universal covering space \(\hat K(G,1)\) is contractible.

    It is also called \emph{the classifying space} or \emph{Eilenberg-Maclane space}.
\end{definition}

\begin{definition}[Aspherical Space]
    \label{def:aspherical-space}
    If \(X\)  is a connected space such that \(\tilde X\) is contractible, then we say \(X\) is a \emph{\(K(\pi,1)\)} (i.e. \(K(\pi_1(X),1)\)). We call such \(X\) \emph{aspherical}.
\end{definition}

\begin{theorem}[Hatcher 1.B.7-9]
    \label{thm:existence-uniqueness-of-KG1}
    For any group \(G\), there exists a (CW) \(K(G,1)\) which is unique up to homotopy equivalence.
    In fact, for any two groups \(G,H\), there exists a bijection
    \[ 
        \hom (G,H) \xleftrightarrow{}
    \]
\end{theorem}

