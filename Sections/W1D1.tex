% !TeX root = ../main.tex
\section{Aspherical Spaces and Sample Early Theorems}
\label{sec:W1D1}

In this course we will explore the relationship between aspherical spaces and (discrete) groups. A convention: unless otherwise stated, all spaces in this course will be CW complexes.

\begin{definition}[\(K(G,1)\)]
    \label{def:KG1}
    Let \(G\) be a group.
    A \emph{\( K(G,1) \) }is a connected space such that \(\pi_1(K(G,1)) \cong G\)  and  whose universal covering space \(\hat K(G,1)\) is contractible.

    It is also called \emph{the classifying space} or \emph{Eilenberg-Maclane space}.
\end{definition}

\begin{definition}[Aspherical Space]
    \label{def:aspherical-space}
    If \(X\)  is a connected space such that \(\tilde X\) is contractible, then we say \(X\) is a \emph{\(K(\pi,1)\)} (i.e. \(K(\pi_1(X),1)\)). We call such \(X\) \emph{aspherical}.
\end{definition}

\begin{theorem}[{\cite[Example 1.B.7, Theorem 1.B.8, Proposition 1.B.9]{hatcher_algebraic_2001}}]
    \label{thm:existence-uniqueness-of-KG1}
    For any group \(G\), there exists a (CW) \(K(G,1)\) which is unique up to homotopy equivalence.
    In fact, for any two groups \(G,H\), there exists a bijection 
    \[
        \hom (G,H) \xleftrightarrow{} \{ \text{based maps } K(G,1) \to K(H,1) \} / \text{based homotopy}
    \]
\end{theorem}


Let \(X\) be a space and \(x_0\) a point in \(X\).
Recall that \(\pi_0(X)\) represents the path components of space \(X\). For \(n\geq 1\), \(\pi_n(X,x_0)\) is the group of based homotopy classes of maps \((\mb{S}^n , \star) \to (X,x_0)\).

For \(n\geq 2\), any map \((\mb{S}^n , \star) \to (X,x_0)\) lifts to \((\mb{S}^n , \star) \to (\tilde X, \tilde x_0)\) (Since \(\pi_1(\mb{S}^n ,\star)\) is trivial). If \(X\) is aspherical, then \((\mb{S}^n , \star) \to (\tilde X, \tilde x_0)\) is null-homotopy. Hence \(\pi_n(X,x_0) =0\). This proves one direction of \cref{thm:aspherical-criterion-trivial-homotopy-group}. The other direction is harder, and we leave it for future discussion.                   

\begin{theorem}[Whitehead’s Theorem, {\cite[Theorem 4.5]{hatcher_algebraic_2001}}]
    \label{thm:aspherical-criterion-trivial-homotopy-group}
    A CW complex \(X\) is aspherical if and only if it's connected and for all \(n\geq 2\), \(\pi_n(X,x_0)=0\).
\end{theorem}

\begin{example}[Aspherical Spaces]
    \label{ex:aspherical-spaces}

    \begin{enumerate}
        \item A point.
        \item \(\mb S^1\).
        \item Products of aspherical spaces (corresponding to direct products of groups).
        \item Wedges of aspherical spaces (corresponding to free products of groups).
        \item Graphs (1-dimensional CW complexes).
        \item Closed orientable surfaces with genus \(\geq 1\).
        \item Complete Riemannian manifolds of non-positive sectional curvature (Cartan-Hadamard theorem).
        \item Complete locally \CATz spaces (A variant of Cartan-Hadamard theorem. See for example \cite[Theorem II.4.1]{bridsonMetricSpacesNonpositive1999}).
    \end{enumerate}
\end{example}

\begin{example}[Non-aspherical Spaces]
    Connect sums typically produce non-aspherical spaces.
\end{example}

\begin{question}
\begin{itemize}
    \item What kind of \(K(G,1)\) spaces does [insert any interesting group] have?
    \item For [insert any interesting group], which \(K(G,1)\) is ``better''?
    \item What do properties of \(K(G,1)\) say about \(G\)?
\end{itemize}
\end{question}

Below we list some sample early theorems without proof.

\begin{theorem}
    \label{thm:fg-criterion-KG1}
    A group \(G\) is finitely generated if and only if \(G\) has a \(K(G,1)\) with a finite 1-skeleton.
\end{theorem}

\begin{theorem}
    \label{thm:fp-criterion-KG1}
     A group \(G\) is finitely presented if and only if \(G\) has a \(K(G,1)\) with a finite 2-skeleton.
\end{theorem}

\begin{theorem}
    \label{thm:torsion-criterion-KG1}
    If \(G\) has a finite-dimensional \(K(G,1)\), then it's torsion free.
\end{theorem}

Observe that those theorems all related to some sort of finiteness of \(K(G,1)\). We introduce some notions of finiteness below.

\begin{definition}
    \label{def:finite-type-KG1}
    A group \(G\) is of \emph{type \(F_n\)} if it has a \(K(G,1)\) with finite \(n\)-skeleton.

    A group \(G\) is of \emph{type \(F_\infty\)} if it's of type \(F_n\) for all \(n\).
    
    A group \(G\) is of \emph{type \(F\)} if it has a finite \(K(G,1)\).

    A group \(G\) has \emph{finite geometric dimension} if it has a finite-dimensional \(K(G,1)\). The \emph{geometric dimension} is the minimal dimension of all of its \(K(G,1)\). 
\end{definition}

It is sometimes difficult to understand the \(K(G,1)\) of a group. We thus use multiple invariance to help determine those types. If \(G\) is a group and \(X\) is its \(K(G,1)\), then the algebraic topological invariance of \(X\) is invariant of \(G\). In particular, we look at the cohomology groups (with trivial coefficients):
\[ H^*(G,\Z):= H^*(X,\Z) \]