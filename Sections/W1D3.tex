% !TeX root = ../main.tex
\section{Compactly supported cellular cohomology}
\label{sec:W1D3}

In this course we focus on \(\Z G\)-module with trivial coefficient. For coefficients other than \(\Z\), a key phrase is ``local systems of coefficients on \(X\)''.

\begin{theorem}
    \label{thm:group-cohomology-cong-cellular-cohomology}
    If \(X\) is a \(K(G,1)\), then for all \(i\geq 0\), \(H^i(G;\Z) \cong H^i(X;\Z)\).
\end{theorem}
\begin{proof}
    Recall that \(X\) gives a free \(\Z g\) resolution of \(Z\) (See \cref{rmk:free-resolution-arises-naturally}):
    \[
        \cdots \to C_1(\tilde X) \xrightarrow{d_1} C_0(\tilde X) \xrightarrow{\epsilon} \Z \to 0.
    \]

    Apply \(\Hom{\_}{\Z }{\Z G}\) to the resolution and consider \(\Hom{C_i(\tilde X  )}{\Z }{\Z G}\). Recall that \(\Z G\) homomorphism has to be invariant under \(G\), so the map has to assign a same number to each \(i\)-cell in the same \(G\)-orbit. It is naturally equivalent to assign numbers to \(i\)-cells in \(X\) i.e. cellular chains in \(X\).
\end{proof}

\begin{definition}
    \label{def:compact-homomorphism}
    Let \(M\) be a \(\Z G\)-module.     Define \(\Hom{M}{\Z }{c} \subseteq \Hom{M }{\Z }{} \) to be 
    \[
        \{ f:M\to \Z \mid \forall m \in M, \text{ for all but finitely many } g\in G, f(gm)=0 \}.
    \] 
\end{definition}

\begin{lemma}[{\cite[Lemma VIII.7.4]{brown_cohomology_1982}}]
    \label{lem:compact-homomorphism-cong-ZG-homomorphism}
    Let \(\G\) be a \(\Z G\)-module, then 
    \[
        \Hom{M }{\Z G }{\Z G} \cong \Hom{M }{\Z }{c}.
    \]
\end{lemma}
\begin{proof}
    Take \(F \in \Hom{M }{\Z G }{\Z G}\). For \(m\in M\), \(F(m)\) can be written as \(\sum_{g\in G} f_g(m) g\) where \(f_g(m)=0\) for all but finitely many \(g\).

    Verify that   \(F \in \Hom{M }{\Z G }{\Z G}\) if and only if \(f_g(m) = f_1(g^{-1}m)\). So \(F\mapsto f_1\) gives a map whose inverse is:
    \[ f\mapsto \{ m\mapsto \sum_{g\in G} f(g^{-1}m) g\}. \]
\end{proof}


\begin{definition}
    \label{def:compactly-supported-cellular-cohomology}
    Let \(Y\) be a locally compact CW complex. That is for all \(n\)-cell \(\sigma\), there are finitely many \(n+1\) cells whose boundary meets \(\sigma\).
    Let \(C_c^i(Y;\Z) \subseteq C^i(Y;\Z)\) be the set of finitely supported cellular cochains with 
    \[ 
        \d^i_c : C_c^i(Y;\Z) \to C_c^{i+1}(Y;\Z) .
    \]
    The homology of this chain complex, denoted as  \(H^i_c(Y;\Z)\), is the \emph{compactly supported cohomology}. 
\end{definition}

\begin{remark}
    The requirement of locally compactness is to ensure that finitely supported cochains and compactly supported cochains are the same set.
\end{remark}

\begin{theorem}
    \label{thm:finite-KG1-compact-cohomology}
    If \(X\) is a finite \(K(G,1)\), then \[ H^i(G;\Z G) \cong H_c^i(\tilde X;\Z) .\]
\end{theorem}
\begin{proof}
    Recall that by \cref{lem:compact-homomorphism-cong-ZG-homomorphism},
    \begin{align*}
          \Hom{ C_i(\tilde X) }{\Z G }{\Z G}  
         \cong \Hom{C_i(\tilde X) }{\Z }{c} 
         = C_c^i(\tilde X;\Z).
    \end{align*} 
    So they induce they same cohomology (See \cref{def:cohomology-of-group,def:compactly-supported-cellular-cohomology}).
\end{proof}

\begin{remark}
    \label{rmk:compact-support-cohomology-as-direct-limit}
    A more natural definition of compactly supported cohomology arises from the direct limit of relative cohomology groups. Observe that a finitely supported \(i\)-cochain is \(0\) outside some compact set \(K\), so it lies in \(C^i(Y,Y\backslash K;\Z)\). We define:
    \[ H_c^*(Y;\Z) = \varinjlim_{K \text{ compact}} H^*(Y,Y\backslash K;\Z) .\]
\end{remark}

\begin{example}[Compactly supported cohomology of \(\R\)]
    \label{ex:compact-supported-cohomology-R}
    Observe that for a connected infinite CW complex \(Y\), \(H_c^0(Y;\Z )= \ker \d^0_c\), which is the set of finitely supported \(1\)-cochains that are constant. So it has to be trivial. (It would be \(\Z\) if \(Y\) is finite.)

    We calculate the compactly supported cohomology for \(\R\). It's cellulated as an infinite graph, which is contractible, so \(H^0_c(\R;\Z)=0\).  For \(n >2\), since there are no \(n\)-cells, \(H^n_c(\R;\Z)=0\). Consider now \(H^1_c(\R;\Z)\). Since there are no \(2\)-cells, all \(1\)-cochains are \(1\)-cocycles. Therefore \(H^1_c(\R;\Z)=C_c^1(\R;\Z)/\im \d^0_c\).

    Consider \(\im \d^0_c\). Recall that \(\d^0_c(\sigma)(e)=\sigma(v_o)-\sigma(v_t)\) for any edge \(v\) with endpoints \(v_o, v_t\).  Observe that each vertex serves as the starting endpoint of one edge and the terminal endpoint of another edge. So \(sum_{e}\d^0_c(\sigma)e = 0\), i.e. \(\im \d^0_c = 0\). It follows that \(H^1_c(\R;\Z) \cong C^1_c(\R;\Z)\).

    Consider now the homomorphisms \(C^1_c(\R;\Z) \to \Z\), they  have the form:
    \[
         \sum_{i} m_i e_i^* \mapsto \sum_i m_i,
    \]
    where \(e^*_i(e_i)=0\) and it sends everything else to \(0\). 

    We need to show that if \(\sum_{i} m_i e_i^*\) has \(\sum_i m_i = 0\) then \(\sum_{i} m_i e_i^* \) is a coboundary. The idea is one can modify a cochain by a coboundary consecutively until one get something supported on a single edge whose coefficient sum is zero. So it's a zero coboundary. For example, applying \( \d(v_i^*)\) to \(e_1 + e_2\) gives \((1-1)e_1 + 0 e_2\).

    TODO not quite so sure

\end{example}

\begin{corollary}
    \begin{align*}
        H^i_c(\Z;\Z[\Z]) = \begin{cases}
            \Z & i=1,\\
            0 & \text{otherwise}.
        \end{cases}
    \end{align*}
\end{corollary}

\begin{example}[Compactly supported cohomology of infinite \(4\)-valence tree]
    \label{ex:compact-supported-cohomology-F2}
    Consider the infinite \(4\)-valence tree \(T\) (the universal cover of the wedge of two circles). Similarly, \(H^i_c(T;\Z)=0\) when \(i\neq 1\). For \(H^1_c(T;\Z)\), by \cref{rmk:compact-support-cohomology-as-direct-limit} it is the direct limit of \(H^1(T,T\backslash K;\Z)\) over all compact sets. Consider the long exact sequence of reduced cohomology groups:
    \[
    \tilde H^0(T;\Z)  \to \tilde H^0 (T\backslash K ; \Z) \to \tilde H^1 (T,T\backslash K; \Z) \to \tilde H^1 (T;\Z).
    \]

    Since \(T\) is contractible, \(\tilde H^i(T;\Z) =0 \), hence \(\tilde H^0 (T\backslash K ; \Z) \cong \tilde H^1 (T,T\backslash K; \Z)\). Now \(\tilde H^0 (T\backslash K ; \Z)\) counts the components of \(T\backslash K\), which increases infinitely as the size of \(K\) increases. 
    
    We conclude that  \(\tilde H^1 (T,T\backslash K; \Z)\) is infinitely generated. In fact, it's an infinitely generated free abelian group.
\end{example}

\begin{corollary}
    \begin{align*}
        H^i_c(\mathbb{F}_2;\Z[\mathbb{F}_2]) = \begin{cases}
            D & i=1,\\
            0 & \text{otherwise},
        \end{cases}
    \end{align*}
    where \(D\) is infinitely generated.
\end{corollary}

Observe that the argument of \cref{ex:compact-supported-cohomology-F2} works for any contractible space. That is, in general, \(1\)-cohomology of a group with group-ring coefficients has rank of numbers of ends of the group plus 1.